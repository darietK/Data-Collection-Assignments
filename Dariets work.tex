\documentclass[10pt,letterpaper]{article}
\begin{document}
\title{IMPACTS OF INTERNET AMONG THE YOUTH ON BEHAVIOR CHANGE:\\ 
A CASE STUDY OF HIGH SCHOOL STUDENTS IN SELECTED SECONDARY SCHOOLS IN KAMPALA, UGANDA
}
\author{by KAMUKAMA DARIET  \\ 216012298 \\  16/U/5346/PS}
\maketitle
\section{Introduction }
This research examines the issues of the relation between internet and its impact on behaviour change of the youth. Today, messages can reach audiences and target groups in real time and they can generate changes and tendencies. Crowds are becoming more powerful through technology, because technology has the ability to unite them.
According to Susan Greenfield, an Oxford University researcher in her article The Quest for Identity in the 21st Century, on Daily Mail UK 14th September, 2010, as growing numbers of people discover the potential of the World Wide Web and as they become active parts of it and as technology becomes even more advanced, expanded, accessible and sophisticated, current forms of communication will transform, taking advantage of the crowd sourcing phenomenon.
This research will find out what internet is. What is its use in the lives of the Secondary students and their implications on their behaviour. New information Technology (IT) is almost everywhere and has dramatically altered the way we live. These tools have become valued elements of life in Uganda merely because they opened many doors to youth and allowed them to interact freely and markedly unlike at any other time in history. In Uganda, college schools have been hit by new generations of youth coming from high schools with quite a decent knowledge about information technology and how to use its tools, especially cellular phones and computers. While any technology can be put to good or bad use, depending on the user, many parents have bought their children cellular phones and PCs so they may use them appropriately and effectively, mainly for learning purposes as well as knowing where they are at any time and come to their help if they need it.

\section{Problem statement }
 This study will try to find out the impact that internet has on the youth ‘s behaviour. Technology has many positive aspects but, in the wrong hands, it can become dangerous. For the young people it is experiments to do what they feel is good or exciting to them and the friends and at the same time avoid adult supervision.  Technology brought about internet which is a valuable tool but is somehow misused by today ‘s youth. The two main forms that the youth use to access internet are cell phones and PCs the which have brought about major changes in their lifestyle. With the current exposure and easy access that the youth are able to get out of these mediums, this study will establish the impacts it has had on the youth. Issues that are expected to arise out of this research include exposure to problematic materials, online victimization of youth, exposition to unnecessary online marketing and advertising, exposure to dangerous online behaviours, issues of identity theft, the emergence of digital divide and generation gap between parents and the youth. According to Ritchel, Matt in an article, ―Growing up Digital, Wired for Distraction.” on The New York Times. 21 Nov. 2010, others include wastage of time, building of shallow and harmful relationships, and, eventually, causing rather than alleviating, users ‘depression, loneliness, social isolation, and withdrawal among others.
\section{GOAL AND OBJECTIVES OF THE STUDY GOAL:}
 The goal of this research is to address the impact and implications of internet on the Ugandan youth especially those in Secondary schools on the way they are using internet and the consequences of that use on their behaviour. 
\section{Specific Objectives: }
1. To determine how the youth in Uganda use internet in their daily lives. 
2. To determine if the youth in Uganda prefer internet as means of communication as opposed       to traditional methods. 
3. To find out the impacts internet has on behaviour changes among the Ugandan youths.
 4. To determine the risks that comes with use of internet on the Ugandan youth.

\section{ RESEARCH QUESTIONS}
The following research questions guided the study: 
1. How do the youth in Uganda use internet in their daily lives? 
2. Do the youth in Uganda prefer internet as a means of communication as opposed to traditional methods? 
3. What are the impacts of internet on the behaviour change of youths in Uganda? 
4. What are the risks that come with use of internet among the youths in Uganda?

\section{JUSTIFICATION OF THE STUDY}
 It is hoped that the findings of this study will bridge the gap of lack of sufficient information on the effects of internet on the youth and behaviour change. The findings of this study may also be useful to the policy makers in various sectors of the government. For instance, in the educational sector curriculum developers will be informed when developing curriculum for the youth. In the health ministry, it will help doctors especially those dealing with counselling of the youth to know which tools to use to effectively communicate to the youth. The results of the study are likely to influence further scholarly research by other researchers who may be interested in this field of knowledge and initiate appropriate mitigation.
\section{SCOPE OF THE STUDY }
The study seeks to find out the impacts of internet among the youth on behaviour change. While the study recognizes that new interactive technologies have impacts on other age groups outside the youth bracket, and as such this study will limit itself only to the youths in Uganda. The study will focus itself only on secondary schools that are based around Kampala central business district, as opposed to other schools outside the stated realm of orientation or geographical boundary. 
\section{LIMITATIONS OF THE STUDY}
The study is limited by time and financial resources and as result the research will have to source for more financial resources and use alternative means. Since few similar studies have been done especially in institutions of higher learning, there is limited empirical literature on the area of impacts of internet on behaviour change especially in the context of Uganda. Another expected limitation is that the youth might fail to give correct information on the basis of invasion of their privacy. The researcher will explain to them that the study is purely for academic purposes and not motivated by any other interests whatsoever. 
\section{ASSUMPTIONS }
The researcher will basically proceed with assumption that she will be able to locate the respondents and that they will be willing to co-operate and give truthful and sincere answers to the items listed in the questionnaires. 
\section{Conclusion of the study}
 In view of the above summary, it was evident that internet played a major role on behaviour change of the respondents. The youths mostly used internet for communicating between their friends and families, doing research work, entertainment. The fact that internet is part of them especially having been born in this era of emerging technology, most felt that they could not do without it. They depended on it for various positive things such as research and contacts with old friends and getting on the loop of what was happening either in their circles, nationally or internationally.


\end{document}